%--------------------------------------------------------------------
%--------------------------------------------------------------------
% exam-randomizechoices-doc.tex
%
% This is the user's guide for the exam-randomizechoices package, 
% by Jesse op den Brouw
%
% The package itself is in the file exam-randomizechoices.sty.


%%% Copyright (c) 2018
% Jesse E. J. op den Brouw
%
% This work may be distributed and/or modified under the
% conditions of the LaTeX Project Public License, either version 1.3
% of this license or (at your option) any later version.
% The latest version of this license is in
%   http://www.latex-project.org/lppl.txt
% and version 1.3 or later is part of all distributions of LaTeX 
% version 2003/12/01 or later.
%
% This work consists of the files exam-randomizechoices.sty,
% exam-randomizechoices.tex and exam-randomizechoices-doc.tex



%%% Jesse op den Brouw
%%% Department of Electrical Engineering
%%% The Hague University of Applied Sciences
%%% Rotterdamseweg 137, 2628 AL, Delft
%%% Netherlands
%%% J.E.J.opdenBrouw@hhs.nl

% The newest version of this documentclass should always be available
% from my web page: http://www-math.mit.edu/~psh/


%--------------------------------------------------------------------
%--------------------------------------------------------------------

\documentclass[12pt,a4paper]{exam}

\usepackage{mathtools}
%\usepackage{amssymb}

%% Set input encoding to UTF-8
\usepackage[utf8]{inputenc}
%% Use T1 output font encoding
\usepackage[T1]{fontenc}

%% Set page layout
\usepackage[a4paper,left=1.0in,right=1.0in,top=1.0in,bottom=1.4in,footskip=0.5in,showframe]{geometry}

%% Use package enumitem
\usepackage{enumitem}
                   
%% Load Ducth spelling
\usepackage[english]{babel}

%% Nice tables
\usepackage{tabu}

%% Mathematical thniks
\usepackage{mathtools}

%% Colors
\usepackage[x11names]{xcolor}

%% Use the Charter font + math symbols
\usepackage[scaled=0.9]{nimbusmono}
\usepackage{charter}
\usepackage[bitstream-charter]{mathdesign}
%% Use microtype
\usepackage[stretch=10]{microtype}

%% Making captions nicer...
\usepackage[font=footnotesize,format=plain,labelfont=bf,up,textfont=sl,up]{caption}

\usepackage[labelformat=simple,font=footnotesize,format=plain,labelfont=bf,textfont=sl]{subcaption}
\captionsetup[figure]{format=hang,justification=centering,singlelinecheck=off,skip=2ex}
\captionsetup[table]{format=hang,justification=centering,singlelinecheck=off,skip=2ex}
\captionsetup[subfigure]{format=hang,justification=centering,singlelinecheck=off,skip=2ex}
\captionsetup[subtable]{format=hang,justification=centering,singlelinecheck=off,skip=2ex}
%% Put parens around the subfig name (a) (b) etc. Needs labelformat simple
\renewcommand\thesubfigure{(\alph{subfigure})}
\renewcommand\thesubtable{(\alph{subtable})}

\usepackage{listings}
\usepackage{textcomp}

\lstset{ %
  language=[AlLaTeX]TeX,
  basicstyle=\ttfamily,
  numbers=left,
  numberstyle=\tiny\color{gray},
  stepnumber=1,                           
  numbersep=8pt,
  showspaces=false,
  showstringspaces=false,
  showtabs=false,
  frame=lines,
  rulecolor=\color{gray},
  tabsize=4,
  captionpos=b,
  breaklines=true,
  breakatwhitespace=true,
  title=\lstname,
  upquote=true,
  aboveskip=\baselineskip,
  belowskip=-\baselineskip,
  escapeinside={(*}{*)}
}

\usepackage[colorlinks,linkcolor=blue]{hyperref}

\usepackage{titlesec}
\usepackage{titletoc}
\usepackage[titles]{tocloft}
\titleformat{\section}{\fontfamily{phv}\selectfont\large\bfseries}{\thesection}{1em}{}
\titleformat{\subsection}{\fontfamily{phv}\selectfont\bfseries}{\thesubsection}{1em}{}
\titlespacing*{\section}{0pt}{\baselineskip}{\aftersubsection}
\titlespacing*{\paragraph}{0pt}{1.0ex plus 1ex minus .2ex}{1.5em}
\newlength{\aftersubtitle}
\setlength{\aftersubtitle}{1.2\baselineskip}
\newlength{\aftersubsection}
\setlength{\aftersubsection}{\aftersubtitle}
\addtolength{\aftersubsection}{-\baselineskip}
\titlespacing*{\subsection}{0pt}{.8\baselineskip}{\aftersubsection}
\titlespacing*{\subsubsection}{0pt}{.6\baselineskip}{0pt}

%% parskip et al.                                            
\usepackage{parskip}

%%
\usepackage[]{exam-randomizechoices}

% In case we're not using hyperref.sty:
\providecommand{\texorpdfstring}[2]{#1}
% The following can be used in \section commands
% without generating pdf warnings:
\newcommand{\bs}{\texorpdfstring{\char`\\}{}}

\newcommand{\docversion}{0.1}
%\newcommand{\docdate}{November 5, 2017}
\newcommand{\docdate}{Draft: \today}

%--------------------------------------------------------------------
%
% Changes since version 2.4 are described in the comments
% near the beginning of the file exam.cls.
%
%--------------------------------------------------------------------

%\makeindex

\newcommand{\indc}[1]{\index{#1@\texttt{\char`\\#1}}}
\newcommand{\indcsub}[2]{\index{#1@\texttt{\char`\\#1}!#2}}
\newcommand{\indcstart}[1]{\index{#1@\texttt{\char`\\#1}|(}}
\newcommand{\indcstop}[1]{\index{#1@\texttt{\char`\\#1}|)}}

\newcommand{\indt}[1]{\index{#1@\texttt{#1}}}
\newcommand{\indtsub}[2]{\index{#1@\texttt{#1}!#2}}
\newcommand{\indtstart}[1]{\index{#1@\texttt{#1}|(}}
\newcommand{\indtstop}[1]{\index{#1@\texttt{#1}|)}}

%---------------------------------------------------------------------
\newenvironment{example}%
   {\bigskip\filbreak
    \subsubsection{Example:}
   }%
   {}

%---------------------------------------------------------------------
%---------------------------------------------------------------------
%---------------------------------------------------------------------
%---------------------------------------------------------------------

\begin{document}

\title{The \texttt{exam-randomizechoices} package\\[2ex]\normalsize LaTeX package for creating random placed choices in multiple choice \\environments using the \texttt{exam} document class}

\author{Jesse op den Brouw\\
  Department of Electrical Engineering\\
  The Hague University of Applied Sciences\\
  Delft, Netherlamds\\
  \href{mailto:J.E.J.opdenBrouw@hhs.nl}{J.E.J.opdenBrouw@hhs.nl}\\[\bigskipamount]
  Copyright \copyright 2018 Jesse op den Brouw\\
  All rights reserved}

\date{\docdate}

\maketitle

%% We like the metric system...
\vspace*{2cm}

\begin{center}
  \small
  This is the user's guide for version~\docversion{} of the
  \verb|exam-randomizechoices| package. 
\end{center}

\clearpage
\tableofcontents

\clearpage

\section{Introduction}

\subsection{License}
This work may be distributed and/or modified under the
%
\index{Latex Project Public License@\LaTeX{} Project Public License}
%
conditions of the \LaTeX{} Project Public
License\index{license},\index{LPPL} either version~1.3 of this
license or (at your option) any later version.  The latest version
of this license is in \url{http://www.latex-project.org/lppl.txt}
and version 1.3 or later is part of all distributions of \LaTeX{}
version 2003/12/01 or later.

This work has the LPPL maintenance status ``author-maintained''.

This work consists of the files \texttt{exam-randomizechoices.sty},
\texttt{exam-randomizechoices.tex} and \texttt{exam-randomizechoices-doc.tex}

\subsection{Acknowledgements}
The author wishes to thank the developers of the \verb|exam| document
class and the \verb|mcexam| package.

\subsection{Warning}
This package is experimental, so it could possibly break \LaTeX{}
compilation. Use this package with care. Please report any problems
to the author.

Please use a recent version of the \texttt{exam} document class. This
package is tested with version 2.603 which is available in most distributions.
Testing with version 2.604\$beta \$ is planned.


\section{Using the \texttt{exam} class}
The \verb|exam| document class is a powerfull class to create exams
with \LaTeX{}. Both open questions and multiple choice questions are
supported. For multiple choice questions we can differentiate between
enumerated lists or checkbox lists. The class documentation states%
\footnote{See \url{https://ctan.org/tex-archive/macros/latex/contrib/exam}.}:

\begin{quote}
The file \verb"exam.cls" provides the \verb"exam" document class,
which attempts to make it easy for even a \LaTeX{} novice to prepare
exams.  Specifically, \verb"exam.cls" sets the page layout so that
there are one inch margins\index{margins} all around (no matter what
size paper you're using) and provides commands that make it easy to
format questions, create flexible headers and footers, change the
margins, and create grading tables.  In more detail:
\begin{itemize}
\item The class will automatically format and number the questions,
  parts of questions, subparts of parts, and subsubparts of subparts.
\item You can include the point value\index{points} of each question
  (or part, or subpart, or subsubpart), with your choice of having the
  point values printed at the beginning of the text of the question,
  opposite that in the left margin, opposite that in the right margin,
  or in the right margin opposite the end of the question.
\item The class will add up the total points for each question (and
  all of its parts, subparts, and subsubparts) and the total
  points\index{points!total} on each page, and make those totals
  available in macros.
\item You can have the class print a grading table, indexed either by
  question number or by page number.
\item You specify the header in three parts: One part to be left
  justified, one part to be centered, and one part to be right
  justified, and one or all of these can be omitted.
\item The footer is also specified in three parts: Left justified,
  centered, and right justified.
\item The header and footer for the first page can be different from
  the ones used on other pages.
\item Both headers and footers can contain more than one line. To
  accommodate headers and footers with several lines, simple commands
  are provided to enlarge the part of the page devoted to the header
  and/or footer, and these commands can give one amount of space on
  the first page and a different amount of space on all other pages.
\item Macros are defined to enable you to state the total number of
  pages in the exam and to change the
  header and/or footer that appears on the last page of the exam .
\item Macros are defined so that the headers and footers can vary
  depending on whether the current page begins a new question or
  continues a question that started on an earlier page (and, if one
  continues onto the current page, to say what the number of that
  question is).  Macros are also defined so that the headers and
  footers can vary depending on whether a question is complete on the
  current page or continues on to the next page (and, if one
  continues, to say what the number of that question is).
\item You can have a horizontal rule at the base of the header and/or
  at the top of the footer.
\item The exam can begin with one or more cover pages, which are
  numbered separately from the main pages of the exam and which can
  have headers and footers different from the ones in the main pages
  of the exam.
\item You can include solutions in your \LaTeX{} file and have these
  solutions either printed or ignored (or replaced automatically by
  space in which the students can write their answers) depending on a
  single command.
\end{itemize}
\end{quote}

Furthermore, not stated in this excerpt, you can typeset bonus questions
with bonus points and grading tables with bonus points.

You can load the \texttt{exam} document class the usual way. An example
might be:

\begin{lstlisting}
\documentclass[a4paper,12pt,addpoints]{exam}
\end{lstlisting}

which loads the \texttt{exam} document class. The paper size is set to
A4 (297 mm $\times$ 210 mm, 11.7~in $\times$ 8.3 in), the document is
typeset with 12 points letters and the question points are added.


\subsection{Typesetting multiple choice questions}
Then, within the document body and between \texttt{\bs begin\{questions\}} and
\texttt{\bs end\{questions\}}, you enter the questions. Only multiple choice
questions are considered here. The \texttt{exam} document class provides four types
of multiple choice question environments:

\begin{description}[labelindent=2ex]
\item[\texttt{choices}] The given choices are typeset in a linear, vertical list.
Each given choice is prepended with a label name which can be set to uppercase
letter, lowercase letter, Roman numerals (uppercase and lowercase) and the Greek
alphabet\footnote{Provided by the \texttt{exam} document class.}.

\item[\texttt{oneparchoices}] The given choices are typeset in a linear, horizontal list.
Long lists are split over multiple lines.
Each given choice is prepended with a label name which can be set to uppercase
letter, lowercase letter, Roman numerals (uppercase and lowercase) and the Greek
alphabet.

\item[\texttt{checkboxes}] The given choices are typeset in a  linear, vertical list.
Each given choice is prepended with a checkbox, which defaults to an open circle.

\item[\texttt{oneparcheckboxes}] The given choices are typeset in a  linear, horizontal
list. Long lists are split over multiple lines.
Each given choice is prepended with a checkbox, which defaults to an open circle.

\end{description}

Examples of the four environment are given below.

\subsection{The \texttt{choices} environment}

An example of a question with the \texttt{choices} environment is:

\begin{lstlisting}
\question[5] What is the result of $1+1$?.

\begin{choices}
\choice 1
\CorrectChoice 2
\choice 3
\choice 4
\end{choices}
\end{lstlisting}

which is typeset as:

\begin{questions}
\setcounter{question}{0}
\question[5] What is the result of $1+1$?.

\begin{choices}
\choice 1
\CorrectChoice 2
\choice 3
\choice 4
\end{choices}
\end{questions}

\subsection{The \texttt{oneparchoices} environment}

An example of a question with the \texttt{oneparchoices} environment is:

\begin{lstlisting}
\question[5] What is the result of $2+2$?..

\begin{oneparchoices}
\choice 1
\choice 2
\choice 3
\CorrectChoice 4
\end{oneparchoices}
\end{lstlisting}

which is typesets as:

\begin{questions}
\setcounter{question}{1}
\question[5] What is the result of $2+2$?..

\begin{oneparchoices}
\choice 1
\choice 2
\choice 3
\CorrectChoice 4
\end{oneparchoices}
\end{questions}

Furthermore, the \texttt{exam} document class provides for two way of typesetting
checkbox questions.

\subsection{The \texttt{checkboxes} environment}

An example of a vertically aligned checkbox environment is:

\begin{lstlisting}
\question[5] What is the result of $1+2$?..

\begin{checkboxes}
\choice 1
\choice 2
\CorrectChoice 3
\choice 4
\end{checkboxes}
\end{lstlisting}

which typesets to:

\begin{questions}
\setcounter{question}{2}
\question[5] What is the result of $1+2$?..

\begin{checkboxes}
\choice 1
\choice 2
\CorrectChoice 3
\choice 4
\end{checkboxes}
\end{questions}

\subsection{The \texttt{onecheckboxes} environment}

The \texttt{oneparcheckboxes} environment typesets the choices
in al linear, horizontal way. An example is given below:

\begin{lstlisting}
\question[5] What is the result of $1+2$?..

\begin{oneparcheckboxes}
\choice 1
\choice 2
\CorrectChoice 3
\choice 4
\end{oneparcheckboxes}
\end{lstlisting}

which typesets to:

\begin{questions}
\setcounter{question}{3}
\question[5] What is the result of $1+2$?..

\begin{oneparcheckboxes}
\choice 1
\choice 2
\CorrectChoice 3
\choice 4
\end{oneparcheckboxes}
\end{questions}

Other types of questions are not considered here.

Please note the use of the \texttt{\bs choice} and \texttt{\bs CorrectChoice}
command. A \texttt{\bs choice} typesets as an item in the list with no special markup.
A \texttt{\bs CorrectChoice} typesets an item in the list with no special markup if
the \texttt{exam} document class option \texttt{answers} is \emph{not} given, and
typesets with special markup if the class option \texttt{answers} is given. This special
markup defaults to boldface for the \texttt{choices} and \texttt{oneparchoices}
environments:

\begin{lstlisting}
\question[5] What is the result of $2+2$?..

\begin{choices}
\choice 1
\choice 2
\choice 3
\CorrectChoice 4
\end{choices}
\end{lstlisting}

which is typesets as:

\printanswers

\begin{questions}
\setcounter{question}{4}
\question[5] What is the result of $2+2$?..

\begin{choices}
\choice 1
\choice 2
\choice 3
\CorrectChoice 4
\end{choices}
\end{questions}

Note that item D is typeset in boldface. The \texttt{checkboxes} and \texttt{oneparcheckboxes}
environments use a checkmark:

\begin{lstlisting}
\question[5] What is the result of $1+2$?..

\begin{oneparcheckboxes}
\choice 1
\choice 2
\CorrectChoice 3
\choice 4
\end{oneparcheckboxes}
\end{lstlisting}

which typesets to:

\begin{questions}
\setcounter{question}{5}
\question[5] What is the result of $1+2$?..

\begin{oneparcheckboxes}
\choice 1
\choice 2
\CorrectChoice 3
\choice 4
\end{oneparcheckboxes}
\end{questions}

\noprintanswers

\section{The package}

\subsection{New multiple choices environments}
Basically, this package provides the user with four new multiple choices
environments:

\begin{description}[labelindent=2ex]
\item[\texttt{randomizechoices}] This is the randomizing counterpart of the
\texttt{choices} environment. It typesets the given items in a random order.

\item[\texttt{randomizeoneparchoices}] This is the randomizing counterpart of the
\texttt{oneparchoices} environment. It typesets the given items in a random order.

\item[\texttt{randomizecheckboxes}] This is the randomizing counterpart of the
\texttt{checkboxes} environment. It typesets the given items in a random order.

\item[\texttt{randomizeoneparcheckboxes}] This is the randomizing counterpart of the
\texttt{oneparcheckboxes} environment. It typesets the given items in a random order.

\end{description}

\subsection{Using the new environments}
We will discuss the \texttt{randomizechoices} environment only. The other
environment work alike.

When using the \texttt{randomizechoices} environment

\begin{lstlisting}
\question[5] What is the result of $1+1$?.

\begin{randomizechoices}
\choice 1
\CorrectChoice 2
\choice 3
\choice 4
\end{randomizechoices}
\end{lstlisting}

which \emph{possibly} is typeset as:

\begin{questions}
\setcounter{question}{6}
\question[5] What is the result of $1+1$?.

\begin{randomizechoices}
\choice 1
\CorrectChoice 2
\choice 3
\choice 4
\end{randomizechoices}
\end{questions}

Here we can see that the resulting output is typeset in another order
then the choices are given. We say \emph{possibly} because the output
depends on the state of the pseudo random generator. See Section~\ref{sec:seeding}

\subsection{Loading the package}
The package is loaded using the well-known \texttt{\bs{usepackage}} command:

\begin{lstlisting}
\usepackage[(* \emph{\textrm{option list}} *)]{exam-randomizechoices}
\end{lstlisting}

The options in \emph{\textrm{option list}} can be any combination of:

\begin{description}[labelindent=2ex]
\item[\texttt{debug}] This option cause the package to emit a lot of debug messages
in the log file. The messages are written to the log by the \texttt{\bs{PackageWarning}}
command. Most IDE's such as TeXMaker will display the messages in the transcript pane.
Debug is off by default. There is no \texttt{nodebug} option.

\item[random] This option globally turns on the randomizing of the choices given for all
available typesetting environments. Random is turned on by default.

\item[norandom] This option globally turns off the randomizing of the choices for all
available typesetting environments. This option is usefull for inspecting the resulting
PDF output file with typesetting the choices in the order they were entered.

\end{description}

\subsection{Package options}

\subsection{Seeding rhe pseudo random generator}
\label{sec:seeding}
To get a consistent randomization, you must seed the pseudo random generator
with the same seed every time you compile your document. You can set the
seed using the \texttt{\bs setrandomizerseed} macro. The macro has a mandatory
argument that is a integer between 0 and $2^{31}-1$, \TeX' largest integer.
Internallu, the PGF macro \texttt{\bs pgfmathsetseed} is called, and it is
flagged that you applied a seed. If you fail to do so, the seeding value is
\texttt{\bs time}$\times$\texttt{\bs year} as stated by the PFG manual%
\footnote{Version 3.0.1a, page 940.}. Some \LaTeX\ compilers keep track of
time by an integer that holds the seconds since midnight. The integer is
incremented every time the time passes a minute boundary. So the scenario
can be that you compile your document a couple of times with no apparent
differences between runs. But if the time passes a minute boundary, the
next time you compile your document you'll see that the environment items
have been rearranged.


\section{Printing the key table}
The package provides the typesetting of a basic key table in vertical direction.
Please note that only the environments \texttt{randomizechoices},
\texttt{randomizeoneparchoices}, \texttt{choices} (if overloaded) and
\texttt{oneparchoices} (if overloaed) can have keys in the key table.
The \texttt{randomizecheckboxes}, \texttt{randomizeoneparcheckboxes},
\texttt{checkboxes} and \texttt{oneparcheckboxes} environments can't 
have keys because of the typesetting regime used by the \texttt{exam}
document class. The \texttt{*choices} environments are typeset as lists
using \texttt{\bs item} which can be provided with a \texttt{\bs label}.
The \texttt{*checkboxes} environment are typeset by skips et al. so 
labeling them would lead to a reference of the current question (or part,
or sub part or sub-sub part). Labeling the correct choices (with the
\texttt{\bs CorrectChoice} command is automaticly handled by the
package.

At the end of the exam, issue the command:

\begin{lstlisting}
\printkeytable
\end{lstlisting}

to print the key table. If an environment can't be labeled, the table entry
will contain \textbf{??}, otherwise it will contain the used typeseting scheme
(which is \texttt{\bs Alph} by default).
The key table is typeset using the \texttt{tabular}
environment, so it can be wrapped in a \texttt{table} environment.

\subsubsection*{Note on the external key table file}
The key table is typeset using an external file, called \texttt{\bs jobname.keytable},
where \texttt{\bs jobname} is the name of your \LaTeX{} file you are compiling. Using
an external file is far more easy then typesetting it directy from the package. See
\url{https://tex.stackexchange.com/questions/367979/latex-foreach-in-tabular-environment}
why.

The key table file can safely be deleted. It is generated each time the
\texttt{\bs printkeytable} command is executed. Edits to the file are lost.

\printkeytable

\end{document}